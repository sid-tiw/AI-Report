\documentclass[journal, compsoc]{IEEEtran}

\usepackage{graphicx}
\usepackage{algorithm}
\usepackage{algpseudocode}
\usepackage{amsmath}
\usepackage{amsfonts}
\usepackage{hyperref}
\usepackage{mathtools}

\newcommand\Mycomb[2][^n]{\prescript{#1\mkern-0.5mu}{}C_{#2}}

\title{Artificial Intelligence Report on Solutions to Laboratory Problems}

\author{Siddhartha~Tiwari, Siddharth~Mani~Tiwari, Saurabh~Kumar, Pushkar~Tiwari}

%\author{
%\IEEEauthorblockN{Siddhartha Tiwari}\IEEEauthorblockA{201851127}
%\and
%\IEEEauthorblockN{Siddharth Mani Tiwari}\IEEEauthorblockA{201851126}
%\and
%\IEEEauthorblockN{Saurabh Kumar}\IEEEauthorblockA{201851113}
%\and
%\IEEEauthorblockN{Pushkar Tiwari}\IEEEauthorblockA{201851095}
%}

\begin{document}
\IEEEtitleabstractindextext{%
\begin{abstract}
This report is the part two of discussions on laboratory problems assigned to us by Dr. Pratik Shah. The problems discussed are from various fields including Hidden Markov Models, Markov Random Fields, Hopfield Networks, Unsupervised Learning, N-Arm Bandit e.t.c. For each problem, the most efficient solution is arrived upon gradually, using different techniques taught in the class.
\end{abstract}
}
\maketitle

\begin{thebibliography}{9}
\bibitem{Travelling Salesman Problem}
Travelling Salesman Problem
\\\texttt{https://en.wikipedia.org/wiki/Travelling\_salesman\_problem}

\bibitem{ai} 
Stuart J. Russell and Peter Norvig. 2003.
\\\textit{Artificial Intelligence: A Modern Approach (2nd. ed.).}
Pearson Education.

\bibitem{test_case} 
Travelling Salesman Test Cases,
\\\texttt{http://www.math.uwaterloo.ca/tsp/vlsi/index.html\#XQF131}
\end{thebibliography}


\end{document}

\end{document}
